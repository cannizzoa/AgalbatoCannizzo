\chapter{Introduction}


\section{Purpose}
This is the Requirement Analysis and Specification Document for the project software \mts{}. The goal of this document is to describe the system in terms of goals, requirements, assumptions and design, to define the different classes of users that will interact with it, and to analyze its most typical and most critical use cases.

This document is targeted to the customer's project managers, to evaluate the project specification from a high-level point of view, and to the intended designers, developers, programmers, analysts and testers, to build the actual software and to maintain, integrate and expand it in the future.


\section{Scope}
The aim of this project is to build a software system able to manage and optimize taxi services in a large city, targeted both to the city public transportation management council and to the citizens as customers of the taxi service. The system will allow passengers to use the service in a smart and accessible way, and at the same time will offer a better service thanks to an enhanced management of taxi resources and allocation.

In particular, passengers and taxi drivers will be registered and remembered by the system. Passengers will be able to request a taxi by using the application and taxi drivers will be notified of these requests by the system, that will divide the city in several taxi zones each with its own queue of taxis. Taxi drivers notified of passengers requests can then accept them.

\section{Definitions, acronyms and abbreviations}
\begin{itemize}

\item Passenger: A citizen recognized by the system as a registered user of \mts{}, using the system as a customer does, to call taxis and be served;
\item Logged passenger: A passenger that is currently and correctly logged into the system;
\item Taxi driver:  A citizen recognized by the system as a registered user of \mts{}, using the system as a taxi driver, recognized by the parent company and of verified identity;
\item Logged taxi driver: A taxi driver that is currently and correctly logged into the system;
\item Request for service: A passenger uses the mobile app or web application to send the central management system a formal and well-formed request to be served by a taxi. This request includes a departure time, an itinerary starting point and ending point and the number of people to be served.
\item Real time request: A request for service that is intended to serve a passenger immediately and is instantly relayed by the system to an available taxi driver;
\item Reservation: A request for service that is not meant to serve a passenger immediately and as such is stored by the system and only relayed to an avaialble taxi driver once the correct time comes;

\end{itemize}


\section{References}
\begin{itemize}

\item Specification Document: My Taxi Service project specification for the Academic Year 2015-16 -- ``Assignments 1 and 2 (RASD and DD).pdf".
\item IEEE Std 830-1998 IEEE Recommended Practice for Software Requirements Specifications.
\end{itemize}


\section{Overview}
This document is structured as follows:

\begin{itemize}

\item Introduction: A declaration of the scope, intent and purpose of this document;
\item Overall description: A high level description of the system, with focus on the goals it will have to meet and on the assumptions that have been held as true during this analisys;
\item Specific requirements: A more detailed look on how the system will need to be implemented, while retaining a sufficiently abstract point of view;
\item System model: A varied representation of the system via its main scenarios identification, an exhaustive listing of UML models, and a consistency analysis made in the Alloy language.

\end{itemize}

