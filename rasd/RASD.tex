\documentclass[a4paper]{report}
\usepackage{hyperref}
\usepackage[dvipsnames]{xcolor}
\usepackage{listings}
\usepackage{booktabs}
\usepackage{caption}
\usepackage{graphicx}

%%%%%%%%%%%%%%%%%%%%%%%%%%%%%%%%%%%%%%%%%%%%%%%%%%%%%%%%
%								Begin alloy language definition
%%%%%%%%%%%%%%%%%%%%%%%%%%%%%%%%%%%%%%%%%%%%%%%%%%%%%%%%

% alloy.sty
% Alloy mode for the LaTeX listings package.
% This is public domain

\lstdefinelanguage{alloy}{
  keywords={%
      assert, pred, all, no, lone, one, some, check, run,
      but, let, implies, not, iff, in, and, or, set, sig, Int, int,
      if, then, else, exactly, disj, fact, fun, module, abstract,
      extends, open, none, univ, iden, seq,
  },
  sensitive=true,  % case sensitive
  morecomment=[l]//,%
  morecomment=[l]{--},%
  morecomment=[s]{/*}{*/},%
  morestring=[b]",
  numbers=left,
  firstnumber=1,
  numberstyle=\tiny,
  stepnumber=5,
  basicstyle=\scriptsize\ttfamily,
  commentstyle=\color{OliveGreen},
  keywordstyle=\bfseries\color{blue},
  ndkeywordstyle=\bfseries,
  tabsize=3,
  breaklines,
  breakindent=20pt,
  breakautoindent,
  showlines=false,
}

% inline
\def\A{%
    \lstinline[language=alloy,basicstyle=\ttfamily,columns=fixed]}

% paragraph
\lstnewenvironment{alloy}[1][]{%
  \lstset{language=alloy,
    floatplacement={tbp},captionpos=b,
    xleftmargin=8pt,xrightmargin=8pt,basicstyle=\ttfamily,#1}}{}

% paragraph from file
\newcommand{\alloyfile}[1]{
  \lstinputlisting[language=alloy,%
    frame=lines,xleftmargin=8pt,xrightmargin=8pt,basicstyle=\ttfamily,columns=fixed]{#1}
}

%%%%%%%%%%%%%%%%%%%%%%%%%%%%%%%%%%%%%%%%%%%%%%%%%%%%%%%%
%								End alloy language definition
%%%%%%%%%%%%%%%%%%%%%%%%%%%%%%%%%%%%%%%%%%%%%%%%%%%%%%%%

\newcommand{\mts}{\textsc{My Taxi Service}}

\lstset{language=Pascal,basicstyle=\ttfamily,keywordstyle=\color{blue}}

\begin{document}

\setcounter{tocdepth}{1}

\title{\Huge Requirement Analisys and Specification Document}
\author{Filippo Agalbato \\ 850481 \and Andrea Cannizzo \\ 790469}
\maketitle

\tableofcontents

\chapter{Introduction}


\section{Purpose}
This is the Requirement Analysis and Specification Document for the project software \mts{}. The goal of this document is to describe the system in terms of goals, requirements, assumptions and design, to define the different classes of users that will interact with it, and to analyze its most typical and most critical use cases.

This document is targeted to the customer's project managers, to evaluate the project specification from a high-level point of view, and to the intended designers, developers, programmers, analysts and testers, to build the actual software and to maintain, integrate and expand it in the future.


\section{Scope}
The aim of this project is to build a software system able to manage and optimize taxi services in a large city, targeted both to the city public transportation management council and to the citizens as customers of the taxi service. The system will allow passengers to use the service in a smart and accessible way, and at the same time will offer a better service thanks to an enhanced management of taxi resources and allocation.

In particular, passengers and taxi drivers will be registered and remembered by the system. Passengers will be able to request a taxi by using the application and taxi drivers will be notified of these requests by the system, that will divide the city in several taxi zones each with its own queue of taxis. Taxi drivers notified of passengers requests can then accept them.

\section{Definitions, acronyms and abbreviations}
\begin{itemize}

\item Passenger: A citizen recognized by the system as a registered user of \mts{}, using the system as a customer does, to call taxis and be served;
\item Logged passenger: A passenger that is currently and correctly logged into the system;
\item Taxi driver:  A citizen recognized by the system as a registered user of \mts{}, using the system as a taxi driver, recognized by the parent company and of verified identity;
\item Logged taxi driver: A taxi driver that is currently and correctly logged into the system;
\item Request for service: A passenger uses the mobile app or web application to send the central management system a formal and well-formed request to be served by a taxi. This request includes a departure time, an itinerary starting point and ending point and the number of people to be served.
\item Real time request: A request for service that is intended to serve a passenger immediately and is instantly relayed by the system to an available taxi driver;
\item Reservation: A request for service that is not meant to serve a passenger immediately and as such is stored by the system and only relayed to an avaialble taxi driver once the correct time comes;

\end{itemize}


\section{References}
\begin{itemize}

\item Specification Document: My Taxi Service project specification for the Academic Year 2015-16 -- ``Assignments 1 and 2 (RASD and DD).pdf".
\item IEEE Std 830-1998 IEEE Recommended Practice for Software Requirements Specifications.
\end{itemize}


\section{Overview}
This document is structured as follows:

\begin{itemize}

\item Introduction: A declaration of the scope, intent and purpose of this document;
\item Overall description: A high level description of the system, with focus on the goals it will have to meet and on the assumptions that have been held as true during this analisys;
\item Specific requirements: A more detailed look on how the system will need to be implemented, while retaining a sufficiently abstract point of view;
\item System model: A varied representation of the system via its main scenarios identification, an exhaustive listing of UML models, and a consistency analysis made in the Alloy language.

\end{itemize}


\chapter{Overall description}

\section{Product perspective}
\mts{} as a software system will take the form of a web application running on a central server with access either through a standard browser or a specific mobile application (for passengers) or simply a mobile application (for taxi drivers) that will allow bidirectional communication, for which a standard HTTPS protocol will be used. Taxi drivers will need to own a smartphone with internet access. There is no software system already in place with which \mts{} will need to be integrated, as this is the case of a greenfield development to modernize a system where everything was previously done "by hand". A programmatic API will be part of the system, to allow for future expansion of its capabilities, such as taxi sharing management. No other specific or apparent constraints on existing hardware and software hold, as long as the system is able to support multiple connections and handle different ROS at a time.

\section{Product functions}
\label{sec:goals}
\mts{} will need to meet the following project goals:
\begin{enumerate}

\item Taxis in the city will be divided in queues and each one will be fairly managed.
\item Passengers will be registered and remembered by the system;
\item Passengers will be able to log in to their personal account;
\item Passengers will be able to request a taxi at their current position or at a position of their choice;
\item Passengers will be able to reserve a taxi in advance;
\item Passengers will be able to access their reservation history;
\item Passengers will be able to modify or delete elements from their reservation history;
\item Passengers will be notified if their request for service is accepted and the ETA of the incoming vehicle;
\item Passengers will be notified if their request for service cannot be met for whatever reason;
\item Taxi drivers will be registered and remembered by the system;
\item Taxi drivers will be able to log in to their personal account;
\item Taxi drivers will be able to inform the system of their availability;
\item Available taxi drivers will be notified when they are called to answer a request for service;
\item Taxi drivers will be able to confirm that they are going to take care of the assigned request for service;
\item Available taxi drivers will be notified by the system that they need to move to a different city zone to ensure a total coverage of possible passenger request;

\end{enumerate}

\section{User characteristics}
\mts{} is intended for two different classes of users: passengers and taxi drivers. As such, it will also offer two different and distinct views and UIs. Passengers will require no special knowledge of the application to use it, for the UI must be designed to be as intuitive and easy-to-use as possible, in such a way that even a child could see it for the first time and understand how it works. Taxi drivers will have access to different functions, but the core principle should well remain the same: there is no requirement for in-depth training (there should not be at all), for the application will simply relay informations and  orders to and from the taxi driver and the system. Being a web-based application, passengers will of couse need internet access to use these services; taxi drivers, on the other hand, are required to own a smartphone with readily available and abundant internet access and, of course, the application installed.

\section{Constraints}
\subsubsection{Regulatory policies}
\mts{} will need to meet any and all requirements imposed by the laws and policies of the city in which it will be deployed, with special regards to the rules concerning taxi services. These will need to be studied in depth, with possible help from the customer's part, and integrated as needed into the management system.

\subsection{Parallel operation}
\mts{} central system will need to constantly monitor the situation of taxis in the whole city in order to dispatch and manage them as best as possibile. At the same time, it will need to be sized appropriately to support the many different connections and operations from clients to the central server, both passengers and taxi drivers.

\section{Assumptions and dependencies}
\begin{enumerate}

\item Users profiles are private and hidden;
\item The system will minimize the risk of not being able to answer to a ROS due to lack of available taxis;
\item All taxi drivers own a suitable smartphone;
\item The mobile app will be available for each major mobile OS and will forward data to the central system in a uniformed way;
\item Taxis will be tracked by a GPS system;
\item The system will receive the GPS reading from each taxi and keep careful track of it;
\item The city will be divided in several zones;
\end{enumerate}
\chapter{Specific requirements}

\section{Functional requirements}
\label{sec:requirements}
This section refers to system goals (specified in~\autoref{sec:goals}). The goal referred to in the requirement description is the one it derives from.

\begin{enumerate}

\item The system must balance taxi drivers workload by assigning each one to a \emph{first-in, first-out} queque in a city zone~(Goal~1);

\item The system must provide a sign up function for passengers~(Goal~2);
\item The system must store all required passenger information~(Goal~2); 

\item The system must provide a log in function to access all passenger features~(Goal~3);

\item The system must provide a function that allows logged passengers to require a taxi~(Goal~4);
\item The system must provide a form in which the logged passenger will be able to add trip information for a request for service (starting point, destination point, number of passengers)~(Goal~4);
\item The system must retrieve GPS information from the logged passenger's mobile phone~(Goal~4);

\item The system must provide a reservation function~(Goal~5);
\item The system must provide a form in which the logged passenger will be able to add trip information for a reservation (starting point, destination point, number of passengers, leaving time)~(Goal~5);
\item The system must finalise a reservation two hours before its requested time, making it unchangeable~(Goal~5);

\item The system must store a reservation history for each passenger~(Goal~6);
\item The system must provide a function which shows reservation history~(Goal~6);

\item The system must provide a function to allow logged passengers to modify a reservation up to two hours before the requested time~(Goal~7);
\item The system must provide a function to allow logged passengers to delete a reservation up to two hours before the requested time~(Goal~7);

\item The system must analyse a newly-made real time request and send a confirmation with the estimated waiting time if it can be fulfilled~(Goal~8);
\item The system must analyse a newly-made reservation request and send a confirmation if it can be fulfilled~(Goal~8);

\item The system must notify passenger if his reservation cannot be met~(Goal~9);
\item The system must delete all reservations in a passenger's history that cannot be met after sending him notice~(Goal~9);

\item The system must provide a sign up function for taxi drivers~(Goal~10);
\item The system must store all required taxi driver information~(Goal~10);
\item The system must check that all taxi drivers trying to sign up own both a valid taxi driver's license and a proper driver's license~(Goal~10);
\item The system must check that the information needed for a taxi driver sign up points to a specific person in the company that has not already registered before confirming~(Goal~10);

\item The system has to provide a log in function to access all taxi drivers features~(Goal~11);

\item The system must provide a function to allow logged taxi drivers to inform the system of their availability~(Goal~12);

\item The system must notify taxi drivers of an incoming request for service~(Goal~13);
\item The system must provide all necessary information for taxi drivers to carry out requests~(Goal~13);

\item The system must provide a function to allow taxi drivers to confirm that they are going to take care of an assigned request for service~(Goal~14);
\item The system must provide an explicit \emph{decline} function to allow taxi drivers to notify that they are not going to take care of an assigned request for service~(Goal~14);
\item The system must flag a taxi driver as having declined an assignment if no answer is received within 30 seconds~(Goal~14);
\item The system must move a taxi driver that has declined an assignment at the end of their queue~(Goal~14);

\item The system must analyse taxi locations and calculate their best possible distribution~(Goal~15);
\item The system must choose which taxi drivers need to be moved to ensure total coverage of the city~(Goal~15);
\item The system must notify taxi drivers in which city zone they have to move, as needed~(Goal~15).
\end{enumerate}


\section{Non-functional requirements}

\subsection{Performance requirements}
\begin{itemize}

\item The system must support a certain number of terminals for taxi drivers: this number will be at first estimated and then extracted from a statistical analysis of the system usage in its first months of life;
\item The system must support a certain number of terminals for passengers: this number will be at first estimated and then extracted from a statistical analysis of the system usage in its first months of life;
\item The system must support the simultaneous use of 50\% of the registered passengers;
\item The system must support the simultaneous use of 100\% of the registered taxi drivers;
\item Taxi drivers notifications must be received in less than 5 seconds.

\end{itemize}

\subsection{Software system attributes}
Reliability is an important factor, as that the system manages all taxis in the city and, were it to fall, the whole taxi circulation would be paralyzed. Portability must be considered top priority as the mobile application will need to run on all major mobile operating systems.

Security is less critical since no confidential client information are registered on or exchanged by the system.


\chapter{System model}

\section{Scenarios}

\subsection{Passenger sign up, real time request, taxi movement}
Bob wants to go to the theater. He downloads the \mts{} app on his smartphone, he opens it and he finds the passenger sign up page. He inserts his name and surname, his address, and his personal citizen ID. After receiving confirmation of a successful operation, he logs in by tapping on the corresponding button. He finds three options: \emph{call a taxi now}, \emph{reserve a taxi for later}, \emph{view reservation history}. He decides to tap the \emph{call a taxi now} function. He finds a form where he inserts the destination address and the number of passengers. The system notifies him to turn his GPS on if he wants to call a taxi to his current location, otherwise he can manually insert a starting address. He taps \emph{confirm} and the system answers by confirming back and sending him the estimated waiting time.

Charlie is a taxi driver already logged in the system, who has given his availability by tapping on the \emph{driver available} button on his personal page in the app. He receives notification that he has been assigned to picking up Bob. He accepts by tapping on the big red \emph{accept} button on his screen, he revs up and goes to the given starting address.

After arriving at destination, Charlie sends Bob on his way and notifies the system that he is available again. The system tells him in which zone of the city he will have to move to keep general widespread taxi coverage.

\subsection{New reservation,  previous reservation modification}
Tom wants to go to the airport at 4.30 AM. Since he is a very anxious person, one week in advance he logs in to \mts{}, chooses the reservation function from his personal page and reserves a ride for the right day and time and destination and starting address. He is relieved in seeing that the system has confirmed his reservation.
Three days later, his anxiety growing, he thinks he should change the reservation in order to be at the airport one hour before what he had planned. He logs again in \mts{} and he chooses the \emph{reservation history} function: the system shows him all of his previous reservations, which in this case is just the one he had made three days earlier. He taps the reservation entry and it pops out the option to \emph{modify} or \emph{delete} it. He carefully taps on \emph{modify} (heavens forbid he deletes it) and he is prompted to insert again all necessary information. He sighs in relief when the system confirms again that everything went well.
Unfortunately, two hours before the scheduled arrival of his taxi (Tom is well awake, least he forgets something for the journey), the system sends him notification that his request cannot be fulfilled due to unexpected events arising outside the sphere of influence of the taxi company. Tom is somewhat taken aback, but, after all, he expected nothing less.

\subsection{Driver sign up and availability, assignment rejection, taxi queque handling}
Eve is a taxi driver. She downloads the \mts{} app and signs up as a taxi driver as per company policy. She is prompted to insert her name and surname, her address, her personal citizen ID, her license number and her car's plate number. The system checks that she is indeed a real, active taxi driver and that she is on the company payroll. After receiving confirmation of a successful operation, she logs in by tapping on the corresponding button.

After going to work one morning, she logs and signals her availability by tapping on the corresponding button on her personal page. Since it's the first time in the day, the app firstly interfaces with her car's GPS tracker, and goes then in waiting mode. After a while, her app comes to life and notifies her that she has ten minutes to pick up a passenger at a certain place. She is busy having a fat cream \emph{bombolone} for breakfast, so she declines.

The system receives Eve's reject and moves her at the end of her queue. At the same time, it notifies the next-in-line.

%%%%%%%%%%%%%%%%%%%%%%%%%%%%%%%%%%%%%%%%%
%%%%%%%%%%%%%%%%%%%%%%%%%%%%%%%%%%%%%%%%%

\section{Use cases and UML diagrams}
The following section gathers a collection of use cases and useful UML diagrams to better model the system in terms of main functionalities, flow of events, components, actors, and events.

\begin{figure}
\centering
\includegraphics[width=\textwidth]{tex-images/usecase-passenger}
\caption{Overview of passenger use cases}
\end{figure}

\begin{figure}
\centering
\includegraphics[width=\textwidth]{tex-images/usecase-driver}
\caption{Overview of taxi driver use cases}
\end{figure}


\begin{table}
\begin{center}
\begin{tabular}{lp{0.8\textwidth}}
\toprule
\textbf{Name}		&	\begin{itemize}
					\item Sign Up
					\end{itemize}	\\
\textbf{Actors}		&	\begin{itemize}
					\item Guest
					\end{itemize}	\\
\textbf{Entry conditions}	&	\begin{itemize}
					\item The guest isn't registered;
					\item The guest has opened the mobile app or the web site home page.
					\end{itemize}	\\
\textbf{Flow of events}	&	\begin{itemize}
					\item	The guest clicks on the \emph{Sign Up} button;
					\item	The system shows him the form to be filled;
					\item	The guest adds all requested information;
					\item	The guest clicks on the \emph{Confirm} button;
					\item	The system verifies the information, confirms the registration and stores new user's information.
					\end{itemize}	\\
\textbf{Exit conditions}	&	\begin{itemize}
					\item	The guest is correctly registered.
					\end{itemize}	\\
\textbf{Exceptions}	&	\begin{itemize}
					\item If the guest is already registered, the system notifies the user with a pop-up message;
					\item	If guest information are incomplete or incorrect, the system notifies the guest and asks him to rewrite them.
					\end{itemize}	\\
\bottomrule
\end{tabular}
\caption{Sign up use case}
\end{center}
\end{table}

\begin{figure}
\centering
\includegraphics[width=\textwidth]{tex-images/sequence-signup}
\caption{Sign up sequence diagram}
\end{figure}



\begin{table}
\begin{center}
\begin{tabular}{lp{0.8\textwidth}}
\toprule
\textbf{Name}		&	\begin{itemize}
					\item Log In
					\end{itemize}	\\
\textbf{Actors}		&	\begin{itemize}
					\item Guest
					\end{itemize}	\\
\textbf{Entry conditions}	&	\begin{itemize}
					\item The guest is registered;
					\item	The guest has opened the mobile app or the web site home page.
					\end{itemize}	\\
\textbf{Flow of events}	&	\begin{itemize}
					\item	The guest clicks on the \emph{Log in} button;
					\item	The system shows him the form to be filled;
					\item	The guest writes user name and password:
					\item	The guest clicks on the \emph{Confirm} button;
					\item	The system verifies the information and shows user's home page.
					\end{itemize}	\\
\textbf{Exit conditions}	&	\begin{itemize}
					\item	The user is correctly logged in.
					\end{itemize}	\\
\textbf{Exceptions}	&	\begin{itemize}
					\item If validation process doesn't terminate well, the system ask the user to rewrite username and password.
					\end{itemize}	\\
\bottomrule
\end{tabular}
\caption{Log in use case}
\end{center}
\end{table}

\begin{figure}
\centering
\includegraphics[width=\textwidth]{tex-images/sequence-login}
\caption{Log in sequence diagram}
\end{figure}


\begin{table}
\begin{center}
\begin{tabular}{lp{0.8\textwidth}}
\toprule
\textbf{Name}		&	\begin{itemize}
					\item Request for service
					\end{itemize}	\\
\textbf{Actors}		&	\begin{itemize}
					\item Passenger
					\item Taxi driver
					\end{itemize}	\\
\textbf{Entry conditions}	&	\begin{itemize}
					\item Passenger is logged into the system.
					\end{itemize}	\\
\textbf{Flow of events}	&	\begin{itemize}
					\item The passenger taps the \emph{Call a taxi now} button;
					\item The passenger inserts a destination address, the number of passengers, and if needed the starting address;
					\item The passenger confirms the request;
					\item The system receives the request;
					\item The system checks the taxi queue of the appropriate city zone and alerts the first taxi driver;
					\item The alerted taxi driver accepts the request and drives to the given address;
					\item The system informs the passenger that all went well and gives him the estimated time of arrival.	
					\end{itemize}	\\
\textbf{Exit conditions}	&	\begin{itemize}
					\item	A taxi driver accepts the request for service.
					\end{itemize}	\\
\textbf{Exceptions}	&	\begin{itemize}
					\item If the notified taxi driver declines, the system chooses the following in queue;
					\item If the request cannot be met for whatever reason, the system notifies the passenger.
					\end{itemize}	\\
\bottomrule
\end{tabular}
\caption{Request for service use case}
\end{center}
\end{table}

\begin{figure}
\centering
\includegraphics[width=\textwidth]{tex-images/sequence-rfs}
\caption{Request for service sequence diagram}
\end{figure}



\begin{table}
\begin{center}
\begin{tabular}{lp{0.8\textwidth}}
\toprule
\textbf{Name}		&	\begin{itemize}
					\item New reservation
					\end{itemize}	\\
\textbf{Actors}		&	\begin{itemize}
					\item Passenger
					\end{itemize}	\\
\textbf{Entry conditions}	&	\begin{itemize}
					\item	The passenger is logged in;
					\item	The system is showing user's home page.
					\end{itemize}	\\
\textbf{Flow of events}	&	\begin{itemize}
					\item	The passenger clicks on the \emph{New reservation} button;
					\item	The system shows him a form to be filled with all trip information (starting address, destination address, leaving date and time, number of passengers);
					\item	The user adds all information and clicks on the \emph{Confirm} button;
					\item	The system checks the information, analyses the request and confirms with a message;
					\item	The system generates a real time request ten minutes before the meeting time.
					\end{itemize}	\\
\textbf{Exit conditions}	&	\begin{itemize}
					\item	The reservation is correctly stored.
					\end{itemize}	\\
\textbf{Exceptions}	&	\begin{itemize}
					\item	If the information is wrong the system asks to rewrite it;
					\item	If the reservation can't be satisfied the system shows an error message.
					\end{itemize}	\\
\bottomrule
\end{tabular}
\caption{New reservation use case}
\end{center}	
\end{table}

\begin{figure}
\centering
\includegraphics[width=\textwidth]{tex-images/sequence-new-reservation}
\caption{New reservation sequence diagram}
\end{figure}



\begin{table}				
\begin{center}
\begin{tabular}{lp{0.8\textwidth}}
\toprule
\textbf{Name}		&	\begin{itemize}
					\item Modify reservation
					\end{itemize}	\\
\textbf{Actors}		&	\begin{itemize}
					\item Passenger
					\end{itemize}	\\
\textbf{Entry conditions}	&	\begin{itemize}
					\item The passenger is logged in;
					\item	The system is showing user's home page.
					\end{itemize}	\\
\textbf{Flow of events}	&	\begin{itemize}
					\item	The passenger clicks on the \emph{Reservation history} button;
					\item	The system shows him all his reservations;
					\item	The user clicks on the reservation he wants to modify;
					\item	The system shows him a pop up window with two options: \emph{Modify} or \emph{Delete};
					\item	The user clicks on the \emph{Modify} button;
					\item	The system shows him a form to be filled with all trip information (starting address, destination address, leaving date and time, number of passengers);
					\item	The user adds all information and clicks on the \emph{Confirm} button;
					\item	The system checks the information, analyses the request and confirms with a message;
					\item	The system generates a real time request at the right moment to satisfy user reservation.
					\end{itemize}	\\
\textbf{Exit conditions}	&	\begin{itemize}
					\item	The reservation is correctly modified and stored.
					\end{itemize}	\\
\textbf{Exceptions}	&	\begin{itemize}
					\item	If there is no reservation in the history, the system shows an error message;
					\item	If it is too late to modify the reservation (less then two hours before the ride), the system shows an error message;
					\item	If the information is wrong the system asks to rewrite it;
					\item	If the reservation can't be satisfied the system shows an error message.
					\end{itemize}	\\
\bottomrule
\end{tabular}
\caption{Modify reservation use case}
\end{center}
\end{table}

\begin{figure}
\centering
\includegraphics[width=\textwidth]{tex-images/sequence-modify-reservation}
\caption{Modify reservation sequence diagram}
\end{figure}



\begin{table}
\begin{center}
\begin{tabular}{lp{0.8\textwidth}}
\toprule
\textbf{Name}		&	\begin{itemize}
					\item Delete reservation
					\end{itemize}	\\
\textbf{Actors}		&	\begin{itemize}
					\item Passenger
					\end{itemize}	\\
\textbf{Entry conditions}	&	\begin{itemize}
					\item	The passenger is logged in;
					\item	The system is showing user's home page.
					\end{itemize}	\\
\textbf{Flow of events}	&	\begin{itemize}
					\item	The passenger clicks on the \emph{Reservation history} button;
					\item	The system shows him all his reservations;
					\item	The user clicks on the reservation he wants to delete;
					\item	The system shows him a pop up window with two options: \emph{Modify} or \emph{Delete};
					\item	The user clicks on the \emph{Delete} button;
					\item	The system deletes the reservation and shows reservation history.
					\end{itemize}	\\
\textbf{Exit conditions}	&	\begin{itemize}
					\item	The reservation is correctly deleted.
					\end{itemize}	\\
\textbf{Exceptions}	&	\begin{itemize}
					\item	If there is no reservation in the history, the system shows an error message.
					\end{itemize}	\\
\bottomrule
\end{tabular}
\caption{Delete reservation use case}
\end{center}
\end{table}

\begin{figure}
\centering
\includegraphics[width=\textwidth]{tex-images/sequence-delete-reservation}
\caption{Delete reservation sequence diagram}
\end{figure}




\begin{table}
\begin{center}
\begin{tabular}{lp{0.8\textwidth}}
\toprule
\textbf{Name}		&	\begin{itemize}
					\item Move taxi driver
					\end{itemize}	\\
\textbf{Actors}		&	\begin{itemize}
					\item Taxi driver
					\end{itemize}	\\
\textbf{Entry conditions}	&	\begin{itemize}
					\item Taxi driver is logged into the system and available.
					\end{itemize}	\\
\textbf{Flow of events}	&	\begin{itemize}
					\item The system checks the total coverage of taxis in the city and decides a taxi driver has to move;
					\item The system notifies the driver they have to move and where;
					\item The taxi driver confirms they are going to move.
					\end{itemize}	\\
\textbf{Exit conditions}	&	\begin{itemize}
					\item	The taxi driver reaches the designated zone.
					\end{itemize}	\\
\textbf{Exceptions}	&	\begin{itemize}
					\item The taxi driver does not move: the system either notifies them again or chooses another one.
					\end{itemize}	\\
\bottomrule
\end{tabular}
\caption{Move taxi driver use case}
\end{center}
\end{table}
\clearpage

\begin{figure}
\centering
\includegraphics[width=\textwidth]{tex-images/sequence-move-taxi}
\caption{Move taxi driver sequence diagram}
\end{figure}




\begin{figure}
\centering
\includegraphics[width=\textwidth]{tex-images/class-diagram}
\caption{Main classes diagram}
\end{figure}

%%%%%%%%%%%%%%%%%%%%%%%%%%%%%%%%%%%%%%%%%
%%%%%%%%%%%%%%%%%%%%%%%%%%%%%%%%%%%%%%%%%

\section{Alloy model}
The following code is the Alloy modelization of the system that has been described in terms of requirements and goals in the previous sections. The \emph{Alloy Analyzer} tool was used to verify these properties, starting from the class diagram. The tool has proven the consistency of this model. In fact, no counterexample has been found and every predicate is consistent. A representation of the world extracted from this model is included.

\begin{figure}
\centering
\includegraphics[width=\textheight, angle=90]{tex-images/alloy-world}
\caption{An alloy world representing this model}
\end{figure}

\alloyfile{alloy/Alloymodel.als}
\appendix
\chapter{Document and work information}

\section{Revisions}
This is the first version of this document. There are currently no revisions.

\section{Tools}

\section{Overall time spent}


\end{document}