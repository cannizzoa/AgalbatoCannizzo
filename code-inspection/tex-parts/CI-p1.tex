\chapter{Overview}

\section{Assignment}
The only class that was assigned is \texttt{ResourcesDeployer.java}, from package \texttt{org.glassfish.resources.module}. From this class, the following methods were assigned for inspection:
\begin{itemize}
\item\ttfamily processArchive(DeploymentContext dc)
\item\ttfamily retainResourceConfig(DeploymentContext dc, Map<String, Resources> allResources)
\item\ttfamily populateResourceConfigInAppInfo(DeploymentContext dc)
\item\ttfamily createResources(DeploymentContext dc, boolean embedded, boolean deployResources)
\item\ttfamily createConfig(Resources resources, Collection<org.\-glassfish.\-resources.\-api.\-Resource> resourcesToRegister, boolean embedded)
\end{itemize}

\section{Functional roles}
As the provided Javadoc for the class makes very clear, it is intended to handle \texttt{glassfish-resources.xml} files bundled in the application, loading and processing them. The assigned methods quite trivially enforce this declaration of purpose.

\section{Code fragments}
\label{sec:code}
Follow here, for the sake of completeness, the actual code fragments that were assigned.

\lstinputlisting[firstnumber=245,firstline=245,lastline=303]{code/ResourcesDeployer.java}

\lstinputlisting[firstnumber=313,firstline=313,lastline=339]{code/ResourcesDeployer.java}

\lstinputlisting[firstnumber=341,firstline=341,lastline=368]{code/ResourcesDeployer.java}

\lstinputlisting[firstnumber=370,firstline=370,lastline=411]{code/ResourcesDeployer.java}

\lstinputlisting[firstnumber=413,firstline=413,lastline=445]{code/ResourcesDeployer.java}
